\documentclass{article}
\usepackage[utf8]{inputenc}
\usepackage{listings}
\usepackage{xcolor}
\usepackage{amsmath}
\usepackage{hyperref}
\usepackage{geometry}
\geometry{margin=1in}

\title{Demonstrating \texttt{super()} Behavior in Python Multiple Inheritance}
\author{}
\date{}

\definecolor{codegray}{gray}{0.95}
\lstset{
    backgroundcolor=\color{codegray},
    basicstyle=\ttfamily\footnotesize,
    frame=single,
    breaklines=true,
    tabsize=4,
    language=Python,
    showstringspaces=false
}

\begin{document}

\maketitle

\section*{Objective}
To demonstrate and understand the behavior of the \texttt{super()} function in Python's multiple inheritance context.

\section*{Python Code Example}
\begin{lstlisting}
class A:
    def process(self):
        print("A.process")
        super().process()

class B:
    def process(self):
        print("B.process")
        super().process()

class C:
    def process(self):
        print("C.process")

class D(A, B, C):
    def process(self):
        print("D.process")
        super().process()

d = D()
d.process()
\end{lstlisting}

\section*{Output}
\begin{lstlisting}
D.process
A.process
B.process
C.process
\end{lstlisting}

\section*{Explanation}

\subsection*{Class Hierarchy}

The inheritance hierarchy is as follows:
\[
\texttt{D} \rightarrow \texttt{A} \rightarrow \texttt{B} \rightarrow \texttt{C}
\]

\subsection*{Method Resolution Order (MRO)}

The MRO determines the order in which Python looks for methods during method calls. For class \texttt{D}, the MRO is:

\begin{verbatim}
(<class '__main__.D'>, <class '__main__.A'>, <class '__main__.B'>, <class '__main__.C'>, <class 'object'>)
\end{verbatim}

\subsection*{How \texttt{super()} Works}

\begin{itemize}
    \item \texttt{d.process()} starts with class \texttt{D}.
    \item \texttt{super().process()} in \texttt{D} calls \texttt{A.process()} (next in MRO).
    \item \texttt{super().process()} in \texttt{A} calls \texttt{B.process()}.
    \item \texttt{super().process()} in \texttt{B} calls \texttt{C.process()}.
    \item \texttt{C.process()} does not call \texttt{super()}, so the chain stops.
\end{itemize}

\section*{Key Takeaways}
\begin{itemize}
    \item \texttt{super()} uses the MRO, not the direct parent class.
    \item It enables cooperative multiple inheritance.
    \item All classes must use \texttt{super()} for the chain to continue.
    \item MRO follows C3 linearization rules.
\end{itemize}

\section*{References}
\begin{itemize}
    \item Python documentation on \texttt{super()}: \url{https://docs.python.org/3/library/functions.html#super}
    \item Real Python guide: \url{https://realpython.com/python-super/}
\end{itemize}

\end{document}
